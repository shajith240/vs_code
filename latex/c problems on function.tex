\documentclass{article}
\usepackage{amsmath}
\usepackage{listings}
\usepackage{xcolor}

\lstset{
    language=C,
    basicstyle=\ttfamily,
    keywordstyle=\color{blue}\bfseries,
    commentstyle=\color{green},
    stringstyle=\color{red},
    showstringspaces=false,
    tabsize=4
}

\title{Programming Questions}
\author{}
\date{}

\begin{document}
\maketitle

\section*{Questions and Solutions}

\begin{enumerate}

    \item \textbf{Write a recursive function to calculate the sum of digits of a number}
    \begin{lstlisting}
    int sumOfDigits(int n) {
        if (n == 0)
            return 0;
        return (n % 10) + sumOfDigits(n / 10);
    }
    \end{lstlisting}

    \item \textbf{Write a function to rotate an array by k positions using pointers}
    \begin{lstlisting}
    void rotateArray(int *arr, int size, int k) {
        k = k % size; // Normalize k
        int temp[k];
        
        // Store first k elements
        for(int i = 0; i < k; i++)
            temp[i] = arr[i];
        
        // Shift remaining elements
        for(int i = k; i < size; i++)
            arr[i-k] = arr[i];
        
        // Place temp elements at end
        for(int i = 0; i < k; i++)
            arr[size-k+i] = temp[i];
    }
    \end{lstlisting}

    \item \textbf{Write a recursive function to find GCD of two numbers}
    \begin{lstlisting}
    int gcd(int a, int b) {
        if (b == 0)
            return a;
        return gcd(b, a % b);
    }
    \end{lstlisting}

    \item \textbf{Write a function that returns a pointer to a dynamically allocated array of prime numbers up to n}
    \begin{lstlisting}
    int* generatePrimes(int n, int* count) {
        int* primes = (int*)malloc(n * sizeof(int));
        *count = 0;
        
        for(int i = 2; i <= n; i++) {
            int isPrime = 1;
            for(int j = 2; j * j <= i; j++) {
                if(i % j == 0) {
                    isPrime = 0;
                    break;
                }
            }
            if(isPrime)
                primes[(*count)++] = i;
        }
        return primes;
    }
    \end{lstlisting}

    \item \textbf{Write a recursive function to reverse a string}
    \begin{lstlisting}
    void reverseString(char* str, int start, int end) {
        if(start >= end)
            return;
        
        char temp = str[start];
        str[start] = str[end];
        str[end] = temp;
        
        reverseString(str, start + 1, end - 1);
    }
    \end{lstlisting}

    \item \textbf{Write a function that takes an array of pointers to strings and sorts them alphabetically}
    \begin{lstlisting}
    void sortStrings(char** strings, int n) {
        for(int i = 0; i < n-1; i++) {
            for(int j = 0; j < n-i-1; j++) {
                if(strcmp(strings[j], strings[j+1]) > 0) {
                    char* temp = strings[j];
                    strings[j] = strings[j+1];
                    strings[j+1] = temp;
                }
            }
        }
    }
    \end{lstlisting}

    \item \textbf{Write a recursive function to check if an array is sorted}
    \begin{lstlisting}
    int isSorted(int* arr, int size) {
        if(size <= 1)
            return 1;
        if(arr[size-1] < arr[size-2])
            return 0;
        return isSorted(arr, size-1);
    }
    \end{lstlisting}

    \item \textbf{Write a function that uses pointers to merge two sorted arrays}
    \begin{lstlisting}
    int* mergeSortedArrays(int* arr1, int size1, int* arr2, int size2) {
        int* result = (int*)malloc((size1 + size2) * sizeof(int));
        int i = 0, j = 0, k = 0;
        
        while(i < size1 && j < size2) {
            if(arr1[i] <= arr2[j])
                result[k++] = arr1[i++];
            else
                result[k++] = arr2[j++];
        }
        
        while(i < size1)
            result[k++] = arr1[i++];
        while(j < size2)
            result[k++] = arr2[j++];
            
        return result;
    }
    \end{lstlisting}

    \item \textbf{Write a recursive function to generate all binary strings of length n}
    \begin{lstlisting}
    void generateBinary(char* str, int n, int i) {
        if(i == n) {
            str[i] = '\0';
            printf("%s\n", str);
            return;
        }
        
        str[i] = '0';
        generateBinary(str, n, i + 1);
        str[i] = '1';
        generateBinary(str, n, i + 1);
    }
    \end{lstlisting}

    \item \textbf{Write a function that implements matrix multiplication using pointers}
    \begin{lstlisting}
    void matrixMultiply(int* A, int* B, int* C, int rows1, int cols1, int cols2) {
        for(int i = 0; i < rows1; i++) {
            for(int j = 0; j < cols2; j++) {
                *(C + i*cols2 + j) = 0;
                for(int k = 0; k < cols1; k++) {
                    *(C + i*cols2 + j) += *(A + i*cols1 + k) * *(B + k*cols2 + j);
                }
            }
        }
    }
    \end{lstlisting}

    % More questions and solutions here...

\end{enumerate}

\end{document}
